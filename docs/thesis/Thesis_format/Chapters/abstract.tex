




\begin{center}
    \textbf{\Large Small-Object Detection in Remote Sensing Images and Video}
\end{center}

\vspace{0.05in}

\begin{center}
    \textbf{\large By}
\end{center}

\vspace{0.05in}

\begin{center}
    \textbf{\large Stamatios Orfanos}
\end{center}

\vspace{0.05in}

\begin{center}
    Submitted to the II-MSc “Artificial Intelligence” on 8, 10, 2024, in \\
    partial fulfilment of the \\
    requirements for the MSc degree
\end{center}





\vspace{1cm}

\textbf{\Large Abstract} \\
Object detection in remote sensing images has been a challenging problem for the computer vision research community because the objects in such images have very few
pixels. There have been improvements in the mean Average Precision (mAP) of the models using different architectures. Most of the detection models are becoming more 
complex and bigger, which can cause a problem usually when a detection model is intended for use in a satellite or an Unmanned Aerial Vehicle, since their computation 
resources are limited. The thesis introduces a novel approach that has achieved a significant reduction in computational complexity, specifically a 56\% decrease in 
Giga Floating Point Operations Per Second (GFLOPs). Also the model was able to achieve an improvement on the Unmanned Aerial Vehicle Small Object Detection (UAV-SOD) 
dataset with a 3.1\% mAP improvement, while maintaining an almost identical performance on the last and most complex dataset the Microsoft Common Object in 
COntext (MS COCO) with a 6.5\% difference.

The results demonstrate the effectiveness of the proposed method, providing useful intel in multi-task learning and achieving better than most 
computational efficiency with detection performance for the utilized datasets.



\vspace{3.5in}


\begin{tabular}{l}
    \textbf{Thesis Supervisor:} Ilias Maglogiannis \\
    \textbf{Title:} Professor
\end{tabular}


