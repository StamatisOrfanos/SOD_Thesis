




\begin{center}
    \textbf{\Large Small-Object Detection in Remote Sensing Images and Video}
\end{center}

\vspace{0.05in}

\begin{center}
    \textbf{\large By}
\end{center}

\vspace{0.05in}

\begin{center}
    \textbf{\large Stamatios Orfanos}
\end{center}

\vspace{0.05in}

\begin{center}
    Submitted to the II-MSc “Artificial Intelligence” on XX XX, 2024, in \\
    partial fulfilment of the \\
    requirements for the MSc degree
\end{center}





\vspace{1cm}

\textbf{\Large Abstract} \\
Object detection in remote sensing images has been a challenging problem for the computer vision research community because the objects in such images have very few pixels (10-20 pixels). There have been many improvements in the mean Average Precision (mAP) of the models using different techniques, but all these improvements come at a cost. The detection models are becoming bigger, which can cause a problem especially when a detection model is intended for use in a satellite or an Unmanned Aerial Vehicle, since their computation capabilities are limited. The thesis introduces a novel approach that has achieved a significant reduction in computational complexity, specifically a 32.67\% decrease in Giga Floating Point Operations Per Second (GFLOPs) for the Transformer Prediction Head YOLOv5 (TPH-YOLO) model. Remarkably, on the Aerial Image Tiny Object Detection (AI-TOD) dataset, this optimization also achieves an increase of 6.3\% mAP at 50\% IoU threshold and 2.4\% at the average mAP across IoU thresholds from 50\% to 95\%. The results demonstrate the effectiveness of the proposed method in balancing computational efficiency with detection performance for the utilized datasets.




\vspace{0.05in}

\textbf{Thesis Supervisor:} Ilias Maglogiannis \\
\textbf{Title:} Professor

