\chapter{Discussion}


\subsection{Differences between models}

The implementation and evaluation of the proposed method on the PANet and
TPH-YOLOv5 models produced two different results on the model performances.
It was extended that while the implementation of the proposed method on TPH-
YOLOv5 significantly reduced the computational cost of the model with little loss
in performance or even improvement in performance on a dataset with remote
sensing images, the implementation of the method on PANet did not have as
much improvement in computational cost and the loss in performance was
higher.

This difference in results is observed due to the micro differences in the model
architecture. PANet uses one prediction head for all 4 levels of the architecture
while TPH-YOLO has a separate prediction head for each level. This negatively
affects the model with the proposed method as due to the smaller paths the
update of weights in the early layers of the model is more direct and the loss of
one layer can affect the update of another layer more significantly. Another
difference in the architecture of the two models is the size of the feature maps in
the "neck" part of the models. PANet uses a fixed depth at all levels of the "neck"
while TPH-YOLOv5 has different depths in the feature maps depending on the
level in the feature pyramid. This enables TPH-YOLO to reduce the depth of the
feature maps in the "neck" part using the proposed method while preserving the
ability to preserve essential information. PANet on the other hand, due to the
small depth of the feature maps (256-d) has a reduced ability to preserve essential
information when applying the proposed method.


\subsection{Performance difference between datasets}



In the experiments conducted within the thesis, a difference in the results of the
implementation of the proposed method in TPH-YOLOv5 between the VisDrone
and AI-TOD datasets was observed. More precisely, the proposed method when
tested on the VisDrone dataset showed a decrease in performance compared to
the original TPH-YOLOv5 while in the tests performed on the AI-TOD dataset
TPH-YOLOv5 with the proposed method showed an optimization in performance
compared to the original.
This is most likely due to the specificity of the VisDrone dataset. As shown in
Figure 29, this dataset has many objects that have not been annotated. This leads
the model to make “false” predictions, when in fact its predictions were correct.
A model with a small gradient path is more affected by these “false” predictions
as the error signal from the false predictions does not have to propagate through
many layers before it reaches the early layers. This results in the model being
more conservative in its predictions which can reduce recall. This phenomenon
is not observed in models with longer gradient paths as the error signal has to
propagate through more layers so the early layers are not affected by such "false"
predictions.


